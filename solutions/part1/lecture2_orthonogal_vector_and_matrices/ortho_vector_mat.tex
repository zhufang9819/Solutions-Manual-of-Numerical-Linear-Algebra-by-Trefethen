\documentclass{article}
\usepackage[utf8]{inputenc}
\usepackage{amsmath}
\usepackage{amsthm}
\usepackage{bm}
\usepackage{amsfonts}
\usepackage{titlesec}
\usepackage{hyperref}
\usepackage{color}
\usepackage{booktabs}
\usepackage{amsthm}
\usepackage{listings}
\usepackage{exsheets}

%% Font Policy
% vector, \bm{}
% matrix, \bm{}
% space, \mathbb{}

%%%% Define "Lecture" (from redefining "Chapter") %%%% 
% \titleformat{\chapter}[block]{\LARGE\bfseries}{Lecture~\arabic{chapter}.}{1em}{}[]
%%%% ============================================ %%%%

%%% Set up exercise-solution format %%%
%\SetupExSheets{
	%	counter-format = ch.qu ,
	%	counter-within = chapter
	%}
%%% =============================== %%%

\renewcommand{\vec}[1]{\mathbf{#1}}
\newtheorem*{remark}{Remark}
\newtheorem{theorem}{Theorem}[section]
\newtheorem{definition}{Definition}[section]
\newtheorem{problem}{Problem}
\newtheorem{corollary}{Corollary}[section]
\newtheorem{lemma}{Lemma}[section]
\newtheorem{property}{Property}[section]

\author{Fang Zhu}
\title{Lecture 2 Orthogonal Vectors and Matrices}

\begin{document}
\maketitle

\section{Prerequisite}
    Before giving the solutions, I would like to prove some basic conclusions about this lecture
    \begin{lemma}
        Given an non-singular matrix $\bm{A}$, then $\bm{A}^{-1}$ is unique
        \begin{proof}
        Suppose that we have two inverse matrices $\bm{C}$ and $\bm{B}$ w.r.t $\bm{A}$. By the definition of inverse. 
        \[\bm{B} =\bm{BI} = \bm{B(AC)} = \bm{(BA)C} = \bm{IC} = \bm{C},\]
        and hence we can conclude that $\bm{B} = \bm{C}$.
        \end{proof}
        \label{lem:inv-unique}
    \end{lemma} 
    
    \begin{lemma}
        Given an non-singular matrix $\bm{A} \in \mathbb{R}^{m\times m}$ , its hermitian conjugate $\bm{A}^{*}$ is also non-singular.
        \begin{proof}
            \[\bm{A}^{-1}\bm{A} = \bm{A}\bm{A}^{-1} = \bm{I},\]
        We can apply the hermitian conjugate to both sides of the equation above:
            \[ \bm{A}^{*} (\bm{A}^{-1})^{*} = (\bm{A}^{-1})^{*}\bm{A}^{*} =  \bm{I}\]
        Hence we can get that $\bm{A}^{*}$ is non-singular.
        \end{proof}
        \label{lem:22}
    \end{lemma} 
    
    \begin{lemma}
    Give a non-singular matrix $\bm{A}$ and its hermitian conjugate $\bm{A}^{*}$, we have
    \[(\bm{A}^{*})^{-1} = (\bm{A}^{-1})^{*}\]
    \end{lemma}
    \begin{proof}
    By lemma \ref{lem:22}, $\bm{A}^{*}$ is non-singular and it's clear that the inverse is $(\bm{A}^{-1})^{*}$. However, we can get that $(\bm{A}^{*})^{-1}$ is also the inverse of $\bm{A}^{*}$ by definition.
    Futher, by lemma \ref{lem:inv-unique}, we have
    $$
    (\bm{A}^{*})^{-1} = (\bm{A}^{-1})^{*}
    $$
    which is exactly what we need to prove.
    \end{proof}
    
    \begin{lemma}
    Given two pure imaginary number $di, bi$, then 
    $$
    (1-di) (1-bi) \neq 0.
    $$
    \begin{proof}
    LHS equals
    \begin{equation*}
    1-bd - (b+d)i, \tag{$\star$}
    \end{equation*}
    if $(\star) = 0$, then we have
    $$
    \begin{aligned}
    1 - bd &= 0 \\
    b+d &=0,
    \end{aligned}
    $$
    which means that
    $$
    -b^2 = 1,
    $$
    since $b \in \mathbb{R}$, the equation above cannot be true, and hence $(1-di)(1-bi) \neq 0$.
    \end{proof}
    \end{lemma}

\section{Solutions}
\subsection{Exercise 2.1}
    \begin{proof}
        Without loss of generality, we assume that $\bm{A}$ is upper-triangular. By the \textit{ex. 1.3}, we can conclude that $\bm{A}^{-1}$ is also upper-triangular. It is clear that $\bm{A}^{*} = \bm{A}^{-1}$ since $\bm{A}$ is unitary. Then $\bm{A}^{\star}$ is also an upper-triangular matrix, which is
        $$
        \bm{A}^{\star}_{i,j} = \bar{a}_{ji} = a_{ij} = 0, \quad (\forall i > j),
        $$
        Hence, the matrix $\bm{A}$ is diagonal. The same follows if $\bm{A}$ is lower-triangular.
    \end{proof}

	
\subsection{Exercise 2.3(a)}
    Let $\bm{x}$ be an eigenvector of matrix $\bm{A}$ w.r.t. the eigenvalue $\lambda$, then
    \[\bm{A}\bm{x} = \lambda \bm{x},\]
    multiplying both sides by $\bm{x}^{\star}$, we get that
    \begin{equation*}
    \bm{x}^{\star} \bm{A} \bm{x} = \lambda \bm{x}^{\star}\bm{x} = \lambda \| \bm{x} \|^{2}, \tag{$\spadesuit$}
    \end{equation*}
    then 
    $$
    \lambda = \frac{\bm{x}^{\star} \bm{A} \bm{x} }{ \| \bm{x} \|^{2}} = \frac{\bm{x}^{\star} \bm{A}^{\star} \bm{x} }{ \| \bm{x} \|^{2}} = \frac{(\bm{x}^{\star} \bm{A} \bm{x} )^{\star}}{ \| \bm{x} \|^{2}} = \bar{\lambda},
    $$
    which means that $\lambda$ is real. 
    
\subsection{Exercise 2.3(b)}
    Let $\bm{x}_1, \bm{x}_2$ be  two eigenvectors of the hermitian matrix $\bm{A}$. Denote $\lambda_k $ the eigenvalue w.r.t $\bm{x}_k (k=1,2)$, where $\lambda_1 \neq \lambda_2 $, then
    $$
    \begin{aligned}
    \lambda_2 \bm{x}_1^\star \bm{x_2} &= \bm{x}_1^{\star} \bm{A} \bm{x}_{2}, \\
    \lambda_1 \bm{x}_2^\star \bm{x_1} &= \bm{x}_2^{\star} \bm{A} \bm{x}_{1}.
    \end{aligned}
    $$
    Note that $\bm{A}$ is hermitian, we can get that
    $$
    \lambda_2 \bm{x}_1^{\star} \bm{x}_2 = \bm{x}_1^{\star} \bm{A} \bm{x}_2 = \bm{x}_1^{\star} \bm{A}^{\star} \bm{x}_2 =  (\bm{x}_2^{\star} \bm{A} \bm{x}_1)^{\star} = \lambda_1^{\star} \bm{x}_1^{\star} \bm{x}_2,
    $$
    then
    $$
    (\lambda_2 - \lambda_1^{\star}) \bm{x}_1^{\star} \bm{x}_2 = 0 \Rightarrow \bm{x}_1^{\star} \bm{x}_2 = 0,
    $$
    which is exactly what we need to prove.
    
\subsection{Exercise 2.4}
    Let $\lambda$ be an eigenvalue of $\bm{A}$, and $\bm{x}$ be the eigenvector w.r.t $\lambda$, then we have $\bm{Ax} = \lambda \bm{x}$ and $\| \bm{Ax} \|_2^2 = \| \lambda \bm{x} \|$, which is
    $$
    \bm{x}^{\star} \bm{A}^{\star} \bm{Ax} = \bm{x}^{\star} \| \lambda^\star \lambda \| \bm{x}. 
    $$
    Since $\bm{A}$ is unitary, then
    $$
    \bm{x}^{\star} \bm{A}^\star \bm{A} \bm{x} = \bm{x}^{\star} \bm{Ix} = \bm{x}^{\star} \| \lambda \|_2^2 \bm{x}.
    $$
    Furthermore,
    $$
    x_1^2 + x_2^2 + \cdots + x_n^2 = \| \lambda \|_2^2(x_1^2 + x_2^2 + \cdots + x_n^2),
    $$
    it follows that $\| \lambda \|_2^2 = 1$ since $\bm{x}$ is non-zero vector.
    
\subsection{Exercise 2.5(a)}
     Let $\bm{x}$ be an eigenvector of matrix $\bm{S}$ w.r.t. the eigenvalue $\lambda$, then 
     \[\bm{S}\bm{x} = \lambda \bm{x}.\]
     By the equation $ex.~2.3(\spadesuit)$, we have
    $$
    \lambda = \frac{\bm{x}^{\star} \bm{S} \bm{x} }{ \| \bm{x} \|^{2}} = \frac{\bm{x}^{\star} (-\bm{S}^{\star}) \bm{x} }{ \| \bm{x} \|^{2}} = \frac{-(\bm{x}^{\star} \bm{S} \bm{x} )^{\star}}{ \| \bm{x} \|^{2}} = -\bar{\lambda},
    $$
    then we can get that $\lambda + \bar{\lambda} = 0$, which means that $\lambda$ is purely imaginary.
    
\subsection{Exercise 2.5(b)}
    Assume that $\lambda$ is the eigenvalue of $\bm{S}$, it follows that $1- \lambda$ is the eigenvalue of $1 - \bm{S}$. Since $\lambda$ is purely imagnary number, then by \textit{Lemma 2.1.4}, we have
    $$
    \det{(1-\bm{S})}  = \prod_{i=1}^{n} (1 - \lambda_i) \neq 0,
    $$
    where $\lambda_i, i \in \{1, 2, \cdots, n\}$ are eigenvalues of $\bm{S}$. Hence we can conclude that $1-\bm{S}$ is non-singular. 
    
\subsection{Exercise 2.5(c)}
        Assume that $\bm{Q} = (\bm{I} - \bm{S})^{-1}(\bm{I} + \bm{S})$, then we have
        $$
        \begin{aligned}
        \bm{Q}\bm{Q}^{\star} = (\bm{I}-\bm{S})^{-1} (\bm{I} + \bm{S}) (\bm{I} + \bm{S}^{\star})\left((\bm{I} + \bm{S})^{-1}\right)^{\star},
        \end{aligned}
        $$
        by \textit{Lemma 2.1.3}, we can get that
        $$
        \begin{aligned}
        \bm{Q}\bm{Q}^{\star} &= (\bm{I}-\bm{S})^{-1} (\bm{I} + \bm{S}) (\bm{I} - \bm{S})\left((\bm{I} - \bm{S})^{-1}\right)^{\star}\\
        &= (\bm{I}-\bm{S})^{-1} \textcolor{red}{(\bm{I} + \bm{S})}\textcolor{blue}{(\bm{I} - \bm{S})} (\bm{I} + \bm{S})^{-1} \\
        &= (\bm{I}-\bm{S})^{-1} \textcolor{blue}{(\bm{I} - \bm{S})}\textcolor{red}{(\bm{I} + \bm{S})} (\bm{I} + \bm{S})^{-1} \\
        &= \bm{I}.
        \end{aligned} 
        $$
        Hence, we can conclude that $(1-\bm{S})^{-1}(1-\bm{S})$ is unitary.
    
\subsection{Exercise 2.6}
    \begin{proof}
        If $\bm{A}$ is singular, there exists a vector $\bm{x} \in \mathbb{C}\setminus \{0\}$ such that
        $$
         \bm{Ax} = \bm{x} + \bm{u}\bm{v}^{\star} \bm{x} = 0,
        $$
        then $\bm{x} = -\bm{u}(\bm{v}^{\star} \bm{x})$ where $\bm{v}^{\star} \bm{u}$ is scalar. Let $\bm{x} = t\bm{u} (t \in \mathbb{R})$, then we can get that 
        $$
        t\bm{u} + \bm{u} (\bm{v}^{\star} t \bm{u}) = t\bm{u} (1+ \bm{v}^{\star} \bm{u}) = 0,
        $$
        It follows that $\bm{v}^{\star} \bm{u} = -1$ since $\bm{x} = t\bm{u} \neq 0$. Assume that $\alpha = -1/(1 + \bm{v}^{\star} \bm{u})$, then
        $$
        (\bm{I}+\bm{u} \bm{v}^{\star})(\bm{I} + \alpha \bm{u}\bm{v}^{\star}) = \bm{I}.
        $$
        Note that we have shown that $\bm{v}^{\star}\bm{u} = -1$ is a necessary condition of $\bm{A}$ is singular. For suffciency, we assume that $\bm{v}^{\star} \bm{u} = -1$. Then for any $ t \in \mathbb{C} \setminus \{ 0 \}$, we have
        $$
        \bm{Au} = t\bm{u} + \bm{u}\bm{v}^{\star} t \bm{u} = t\bm{u} + t\bm{u} (\bm{v}^{\star} \bm{u}) = 0,
        $$
        which implies that $\bm{A}$ is singular. Combined, we conclude that $\bm{A}$ is singular iff. $\bm{v}^{\star} \bm{u} = -1$. In this case, 
        $$
        \mathrm{null} (\bm{A}) = \{ t\bm{u}, t \in \mathbb{R} \},
        $$
        the linear subspace spanned by $\bm{u}$.
    \end{proof}
    
\subsection{Exercise 2.7}
    \begin{proof}
    We can verify that $\bm{H}_{k+1}$ is Hadamard matrix directly,
    $$
    \begin{aligned}
    \bm{H}_{k+1}^{T} \bm{H}_{k+1} & = \left(\begin{array}{cc}
        \bm{H}_{k}^T &  \bm{H}_{k}^T\\
         \bm{H}_{k}^T& -\bm{H}_{k}^T
    \end{array}\right)  \left(\begin{array}{cc}
        \bm{H}_{k} &  \bm{H}_{k}\\
         \bm{H}_{k} & -\bm{H}_{k}
    \end{array}\right)\\
    &= \left(\begin{array}{cc}
         \bm{H}_{k}^T \bm{H}_{k} + \bm{H}_{k}^T\bm{H}_{k}& \bm{H}_{k}^T \bm{H}_{k}-\bm{H}_{k}^T \bm{H}_{k} \\
         \bm{H}_{k}^T \bm{H}_{k} - \bm{H}_{k}^T \bm{H}_{k}& \bm{H}_{k}^T \bm{H}_{k}+\bm{H}_{k}^T \bm{H}_{k} 
    \end{array}\right) \\
    &= \left( \begin{array}{cc}
         2\bm{H}_{k}^T \bm{H}_k& \bm{0} \\
         \bm{0}& 2\bm{H}_{k}^T \bm{H}_k 
    \end{array} \right) \\
    &= \left( \begin{array}{cc}
         2c\bm{I}_k & \bm{0} \\
         \bm{0}& 2c\bm{I}_k 
    \end{array} \right) \\
    &= 2c \cdot \bm{I}_{2k}.
    \end{aligned}
    $$
    Then we can get that $\bm{H}_{k+1}^{T} = 2c \bm{H}_{k+1}^{-1}$. Note that the entries of $\bm{H}_{k+1}$ are also all $\pm 1$ by the recursion formula. Hence $\bm{H}_{k+1}$ is also a Hadamard matrix.
    \end{proof}
\end{document}
