\documentclass{article}
\usepackage[utf8]{inputenc}
\usepackage{amsmath}
\usepackage{amsthm}
\usepackage{bm}
\usepackage{amsfonts}
\usepackage{titlesec}
\usepackage{hyperref}
\usepackage{color}
\usepackage{booktabs}
\usepackage{amsthm}
\usepackage{listings}
\usepackage{exsheets}

%% Font Policy
% vector, \bm{}
% matrix, \bm{}
% space, \mathbb{}

%%%% Define "Lecture" (from redefining "Chapter") %%%% 
% \titleformat{\chapter}[block]{\LARGE\bfseries}{Lecture~\arabic{chapter}.}{1em}{}[]
%%%% ============================================ %%%%

%%% Set up exercise-solution format %%%
%\SetupExSheets{
	%	counter-format = ch.qu ,
	%	counter-within = chapter
	%}
%%% =============================== %%%

\renewcommand{\vec}[1]{\bm{#1}}
\newtheorem*{remark}{Remark}
\newtheorem{theorem}{Theorem}[section]
\newtheorem{definition}{Definition}[section]
\newtheorem{problem}{Problem}
\newtheorem{corollary}{Corollary}[section]
\newtheorem{lemma}{Lemma}[section]
\newtheorem{property}{Property}[section]

\author{Fang Zhu}
\title{Lecture 5 More on the SVD}

\begin{document}
	\maketitle
	
\section{Prerequisite}
We first give a more detailed proof of \textit{Theorem 5.2} in the book.
\begin{theorem}
$\mathrm{range} (\bm{A}) = <\bm{u}_1, \cdots, \bm{u}_l>$, $\mathrm{null} (\bm{A}) = <\bm{v}_{l+1}, \cdots, \bm{v}_{n}>$
\end{theorem}

\begin{proof}
Assume that $\bm{A} = \bm{U} \bm{\Sigma} \bm{V}^{\star}$ and $\bm{A}$ has $l$'s non-zero
singular values, then
$$
\begin{aligned}
\mathrm{range} (\bm{A}) & = \{ \bm{Ax}: \bm{x} \in \mathbb{R}^{n}\} \\
& = \{ \bm{U} \bm{\Sigma} \bm{V}^{\star}\bm{x}: \bm{x} \in \mathbb{R}^{n}\} \\
& = \{ \bm{U} \bm{\Sigma}  \bm{y}: \bm{y} = \bm{V}^{\star}\bm{x}, \bm{x} \in \mathbb{R}^{n}\} \\
& = \{ \bm{U} \bm{\Sigma} \bm{y}: \bm{y} \in \mathbb{R}^{n}\}.
\end{aligned}
$$
Let $\bm{c}_j  = \sigma_j \bm{y}_j, (j = 1, 2, \cdots, l)$, it is clear that $\bm{c}_j$ can be any vector in $\mathbb{R}^n$ since $\bm{y}$ is an arbitrary vector in $\mathbb{R}^n$. Hence, we can conclude that
$$
\mathrm{range}(\bm{A}) = \{ \bm{U} \bm{c}: \bm{c} \in \mathbb{R}^n \} = <\bm{u}_1, \bm{u}_2, \cdots, \bm{u}_l>.
$$
Further the null space of $\bm{A}$ is as follows:
$$
\begin{aligned}
    \mathrm{null}( \bm{A} )&= \{ \bm{x} \in \mathbb{R}^n: \bm{Ax} = 0\} \\
    & = \{ \bm{x} \in \mathbb{R}^n: \bm{U} \bm{\Sigma} \bm{V}^{\star}\bm{x} = 0\} \\
    & = \{ \bm{x} \in \mathbb{R}^n: \bm{\Sigma} \bm{V}^{\star}\bm{x} = 0\}.
\end{aligned}
$$
Note that any vector $\bm{x}$ can be expressed by $\bm{x} = \sum_{j}^{n} x_i \bm{v}_j$. Substituting this expression into $\bm{\Sigma} \bm{V}^{\star}\bm{x} = 0$ gives that
$$
(\sigma_1 x_1, \sigma_2 x_2, \cdots, \sigma_l x_l, \sigma_{l+1} x_{l+1}, \cdots, \sigma_n x_n)^T = 0.
$$
Since $\sigma_j > 0$ for $1 \leq j \leq l$, it follows that $c_j = 0$ for these index values. Moreover, $\sigma_j = 0$ for $l+1 \leq j \leq n$ implies that the coefficient $c_j$ for these $j$-values are arbitrary. We can conclude that
$$
\mathrm{null} (\bm{A}) = \{ \bm{x} = \sum_{i=1}^l c_i \bm{v}_i, c_i \in \mathbb{R}\} = <\bm{v}_{l+1}, \bm{v}_{l+2}, \cdots, \bm{v}_n>.
$$
\end{proof}


\section{Solutions}
\subsection{Exercise 5.1}
First we compute the singular values $\sigma_i$ by finding the eigenvalues of
$$
\bm{A}^{\star} \bm{A} = \begin{pmatrix}
1 & 2 \\
2 & 8
\end{pmatrix}.
$$
The characteristic polynomial of $\bm{A}^\star \bm{A}$ is
$$
\mathrm{det} (\bm{A}^{\star} \bm{A} - \lambda \bm{I}) = (1- \lambda) (8- \lambda) - 4 = 8 + \lambda^2 - 9 \lambda - 4 = \lambda^2 - 9 \lambda + 4,
$$
so the singular values are
$$
\sigma_{max} = \sqrt{\frac{ 9 + \sqrt{65} }{ 2}}, \quad \sigma_{min} = \sqrt{\frac{ 9 - \sqrt{65} }{ 2}}.
$$

\subsection{Exercise 5.2}
Give a full SVD of $\bm{A} $ as $\bm{A} = \bm{U} \bm{\Sigma} \bm{V}^{\star} $ if $\mathrm{rank}(\bm{A}) = n$, then the claim is clearly true (sequence as $\{\bm{A}_1 = \bm{A}, \bm{A}_2 = \bm{A}, \cdots, \bm{A}_n = \bm{A}, \cdots \}$. Otherwise, if $\mathrm{rank}(\bm{A})= r < n$, we can define a sequence of $\bm{A}$ as 
$$
\bm{A}_n = \bm{U} \begin{pmatrix}
    \sigma_1&&&&&& \\
    &\sigma_2&&&&& \\
    &&\ddots&&&& \\
    &&& \sigma_r &&&\\
    &&&&\frac{1}{n}&& \\
    &&&&&\ddots& \\
    &&&&&&\frac{1}{n}
\end{pmatrix} \bm{V}^{\star},
$$
then we can easily verify that 
$$
\lim_{n \to \infty} \| \bm{A} - \bm{A}_n \|_2 = 0.
$$

\subsection{Exercise 5.3(a)}
First we compute the singular values $\sigma_i$ by finding the eigenvalues of
$$
\bm{A}^{\star} \bm{A} = \begin{pmatrix}
104 & -72 \\
-72 & 146
\end{pmatrix}.
$$
The characteristic polynomial of $\bm{A}^\star \bm{A}$ is
$$
\mathrm{det} (\bm{A}^{\star} \bm{A} - \lambda \bm{I}) = (\lambda - 200) (\lambda - 50) = 0,
$$
so the singular values are
$$
\sigma_1 = \sqrt{\lambda_1} = 5 \sqrt{2}, \quad \sigma_2 = \sqrt{\lambda_2} = 10 \sqrt{2}.
$$
For $\lambda_1 = 200$, we have
$$
\bm{A}^\star \bm{A} - \lambda \bm{I} = \begin{pmatrix}
    -96 & -72 \\
    -72 & -54
\end{pmatrix},
$$
a unit vector in the kernel of the matrix is $\bm{v}_1 = (-3/5, 4/5)^T$.
For $\lambda_2 = 50$, we have
$$
\bm{A}^\star \bm{A} - \lambda \bm{I} = \begin{pmatrix}
    54 & -72 \\
    -72 & 96
\end{pmatrix},
$$
a unit vector in the kernel of the matrix is $\bm{v}_2 = (4/5, 3/5)^T$. So at this point, we know that
$$
\bm{A}= (\bm{u}_1, \bm{u}_2) \begin{pmatrix}
    -\frac{3}{5} & \frac{4}{5}\\
    \frac{4}{5} & \frac{3}{5}
\end{pmatrix}.
$$
Further we can compute $\bm{u}_1, \bm{u}_2$ by $\sigma_i \bm{u}_i = \bm{A} \bm{v}_i$, which gives that
$$
\bm{u}_1 = \begin{pmatrix}
    \frac{\sqrt{2}}{2}\\
    \frac{\sqrt{2}}{2}
\end{pmatrix},\quad \bm{u}_2 = \begin{pmatrix}
    \frac{\sqrt{2}}{2}\\
    -\frac{\sqrt{2}}{2}.
\end{pmatrix}
$$
So in its full glory the SVD is 
$$
\bm{A} = \begin{pmatrix}
    \frac{\sqrt{2}}{2}& \frac{\sqrt{2}}{2} \\
    \frac{\sqrt{2}}{2} &-\frac{\sqrt{2}}{2}
\end{pmatrix} \begin{pmatrix}
    10\sqrt{2} & 0 \\
    0 & 5\sqrt{2}
\end{pmatrix} \begin{pmatrix}
    -\frac{3}{5} & \frac{4}{5} \\
    \frac{4}{5} & \frac{3}{5}
\end{pmatrix}.
$$
\subsection{Exercise 5.3(b)}
Singular values: $$\sigma_1 = 10\sqrt{2}, \quad \sigma_2 = 5\sqrt{2}$$.
Left singular vectors:
$$
\bm{u}_1 = \begin{pmatrix}
    \frac{\sqrt{2}}{2}\\
    \frac{\sqrt{2}}{2}
\end{pmatrix},\quad \bm{u}_2 = \begin{pmatrix}
    \frac{\sqrt{2}}{2}\\
    -\frac{\sqrt{2}}{2}
\end{pmatrix}.
$$
Right singular vectors:
$$
\bm{v}_1 = \begin{pmatrix}
    \frac{-3}{5}\\
    \frac{4}{5}
\end{pmatrix},\quad \bm{v}_2 = \begin{pmatrix}
    \frac{4}{5}\\
    -\frac{3}{5}
\end{pmatrix}.
$$
\subsection{Exercise 5.3(c)}
We can compute the norms via the definition in \textit{Lecture 3 Norms}.
$$
\begin{aligned}
    \| \bm{A} \|_1 & = \max_{1 \leq j \leq 2} \| \bm{a}_j\| = 16, \\
    \| \bm{A} \|_2 & = \| \bm{U} \bm{\Sigma} \bm{V}^{\star}\|_2 = \| \bm{\Sigma} \|_2 = 10\sqrt{2},\\
    \| \bm{A} \|_F &= \| \bm{\Sigma} \|_F = \sqrt{50 + 200} = 5\sqrt{10}.
\end{aligned}
$$
\subsection{Exercise 5.3(d)}
We can compute $\bm{A}^{-1}$ via SVD as follows:
$$
\begin{aligned}
    \bm{A}^{-1} &= (\bm{U} \bm{\Sigma} \bm{V}^{\star})^{-1} = \bm{V} \bm{\Sigma}^{-1} \bm{U}^{\star} \\
    &= \begin{pmatrix}
        -\frac{3}{5} & \frac{4}{5} \\
        \frac{4}{5} & \frac{3}{5}
    \end{pmatrix} \begin{pmatrix}
        \frac{1}{10\sqrt{2}} & 0 \\
        0 & \frac{1}{5\sqrt{2}} 
    \end{pmatrix}
    \begin{pmatrix}
        \frac{\sqrt{2}}{2} & \frac{\sqrt{2}}{2} \\
        \frac{\sqrt{2}}{2} & -\frac{\sqrt{2}}{2}
    \end{pmatrix}\\
    &= \begin{pmatrix}
        -\frac{1}{20} & -\frac{11}{100} \\
        \frac{1}{10} & -\frac{1}{50}
    \end{pmatrix}.
\end{aligned}
$$
\subsection{Exercise 5.3(e)}
The characteristic polynomial of $A$ is
$$
\det(\bm{A} - \lambda \bm{I}) = \lambda^2 - 3\lambda + 100 = 0,
$$
we can get that
$$
\lambda_1 = \frac{3 + i\sqrt{391}}{2}, \quad \lambda_2 = \frac{3 -i\sqrt{391}}{2}.
$$
\subsection{Exercise 5.3(f)}
We can verify the claim as follows:
$$
\begin{aligned}
\det(\bm{A}) &= -2\times 5 + 10 \times 11 = 100 = \lambda_1 \cdot \lambda_2 \\
|\det(\bm{A})| &= 100 = 10\sqrt{2} \times 5\sqrt{2} = \sigma_1 \cdot \sigma_2 .
\end{aligned}
$$
\subsection{Exercise 5.3(g)}
It is clear that the ellipsoid is a rotation of another ellipsoid $\mathcal{E}$ whose equation are
$$
\frac{x^2}{\sigma_1^2} + \frac{y^2}{\sigma_2^2} = 1
$$
and hence we can compute the area of $\mathcal{E}$ as the area of the original ellipsoid. We first let
$$
y = f(x) = \sigma_2 \sqrt{1 - \frac{x^2}{\sigma_1^2}},
$$
then
$$
\begin{aligned}
\mathrm{Area}(\mathcal{E}) &= 4 \int_{0}^1 f(x) dx \\
&= 4 \int_{0}^1 \sigma_2 \sqrt{1 - \frac{x^2}{\sigma_1^2}} dx \\
&= 4 \int_{0}^{\frac{\pi}{2}} \sigma_2\sigma_1 \cos^2 \theta d\theta, \quad (x = \sigma_1 \sin \theta) \\
& = 4 \int_{0}^{\frac{\pi}{2}} \frac{\sigma_1 \sigma_2}{2}(1+ \cos (2\theta)) d\theta \\
& = 4 \times \frac{\sigma_1 \sigma_2\pi}{4} = \sigma_1 \sigma_2 \pi.
\end{aligned}
$$
Thus the area of the original ellipsoid is $\sigma_1 \sigma_2 \pi = 100 \pi$.

\subsection{Exercise 5.4}
  
Let $\bm{A}$ be an $m \times m$ complex matrix with singular values $\sigma_1, \sigma_2, \cdots, \sigma_m$, and let $\bm{u}_i$ and $\bm{v}_i$ be the corresponding left and right singular vectors. If we define 
\[ \bm{\beta}_i = \begin{pmatrix}
a \bm{v}_i \\
b \bm{u}_i \\
\end{pmatrix} \in \mathbb{C}^{2m}, \]
and the matrix
\[ \bm{B} = \begin{pmatrix}
\bm{0} & \bm{A}^\star \\
\bm{A} & \bm{0} \\
\end{pmatrix}, \]
then we claim that the eigenvalues of $ \bm{B} $ are $ \pm \sigma_1, \pm \sigma_2, \dots, \pm \sigma_m $.
% Then, we have $\bm{A}\bm{v}_i = \sigma_i \bm{u}_i$ and $\bm{A}^{\star} \bm{u}_i = \sigma_i \bm{v}_i$, where $\bm{A}^\star$ is the conjugate transpose of $\bm{A}$ and $\sigma_k \in \mathbb{R}$. We define $\bm{\beta}_i = [a\bm{v}_i, b\bm{u}_i] \in \mathbb{C}^{2m}$ and consider the matrix
% $$
% \bm{B} = \begin{pmatrix}
%     \bm{0} & \bm{A}^\star \\
% \bm{A} & \bm{0}
% \end{pmatrix}
% $$
Consider the product:
$$
\bm{B} \bm{\beta}_i =  \begin{pmatrix}
b \sigma_i \bm{v}_i\\
a \sigma_i \bm{u}_i
\end{pmatrix}
$$
Given that \( \bm{\beta}_i \) is an eigenvector of \( \bm{B} \) with eigenvalue \( \lambda \), we deduce the following systems:

For \( \lambda = \sigma_i \):
\[ \left\{ \begin{array}{l}
a = b \\
b = a \\
\end{array} \right. \]
And for \( \lambda = -\sigma_i \):
\[ \left\{ \begin{array}{l}
a = -b \\
b = -a \\
\end{array} \right. \]
From these systems, it is evident that \( a = b \) and \( a = -b \), respectively. 

To construct the eigen decomposition of \( \bm{B} \), we define the matrix:
$$
\bm{S} = \begin{pmatrix}
a\bm{v}_1, &a\bm{v}_2, &\cdots, &a\bm{v}_m, &a\bm{v}_{1}, &a\bm{v}_{2}, &\cdots,&a\bm{v}_{m} \\
a\bm{u}_1, &a\bm{u}_2, &\cdots,&a\bm{u}_m, &-a\bm{u}_{1}, &-a\bm{u}_{2}, &\cdots, &-a\bm{u}_{m}
\end{pmatrix}
$$
as well as
$$
\bm{\Lambda} = 
 \begin{pmatrix}
   \sigma_{1} &  &  &&&&&\\ 
   & \sigma_{2} &  & &&&&\\ 
   &  &  \ddots & &&&&\\ 
   &  &   & \sigma_{m} &&&&\\
   &  &   & &-\sigma_{1} && &\\
   &  &   & & & -\sigma_{2}&& \\
   &  &   & & & & \ddots &\\
   &  &   & & & & & -\sigma_{m}
 \end{pmatrix},
$$
then there is
$$
\bm{B} \bm{S} = \bm{S} \bm{\Lambda},
$$
But at this time $\bm{S}$ is not necessarily invertible. Next, we make $\bm{S}$ an orthogonal matrix by taking an appropriate $a$, so that $\bm{S}$ is invertible. It is not difficult to calculate, assuming that $\bm{S}$ is orthogonal, that is, $\bm{S}^{\star} \bm{S} = \bm{S} \bm{S}^{\star} = \bm{I}$, then
$$
a^2 + a^2 = 1,
$$
We can get $a = \pm \frac{1}{\sqrt{2}}$, without loss of generality, we make $a = \frac{1}{\sqrt{2}}$, then $S $ is an orthogonal matrix, so it is invertible, then the eigendecomposition of $B$ is
$$
\bm{B} = \bm{S}^{-1} \bm{\Lambda} \bm{S}.
$$





\begin{remark}
In fact, this proof is not perfect. We did not detail how to find the eigenvalue of $\bm{B}$, that is, the root of the characteristic polynomial corresponding to $\bm{B}$ has and only $\pm \sigma_i$, which needs to be improved later. If readers have any good ideas, welcome to contact me.
\end{remark}






\end{document}