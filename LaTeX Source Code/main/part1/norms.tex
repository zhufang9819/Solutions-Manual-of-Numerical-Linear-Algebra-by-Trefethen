\section{Prerequisite}
\begin{definition}[Vector Norm]
    A \textit{norm} is a function $\| \cdot \|: \mathbb{C}^{m} \to \mathbb{R}$ that assigns a real-valued length to each vector. In order to conform to a reasonable notion of length, a norm must 
    satisfy the following three conditions. For all vectors $x$ and $y$ and for all scalars $\alpha \in \mathbb{C}$,
    \begin{enumerate}
        \item[(1)] $\| x \| \geq 0$ and $\| x \| = 0$ only if $x = 0$,
        \item[(2)] $\| x + y\| \leq \| x\| + \| y\|$,
        \item[(3)] $\| \alpha x \| = |\alpha| \|x\|$.
    \end{enumerate}
\end{definition}
\begin{lemma}
Given a permutation matrix $\bf{P} \in \mathcal{M}_{m \times n}$, and a vector $\bm{x} \in \mathbb{C}^{n}$, then 
$$
    \|\bm{Px} \|_p = \|\bm{x} \|_p.
$$
\begin{proof} 
By the definition of vector norm, we can get that
$$
\|\bm{x}\|_p = \begin{cases}
\left(\sum_{i=1}^{m} |x_i|^{p})^{\frac{1}{p}}\right), & 1\leq p < \infty; \\
\max_{i} \{ x_i \}, & p = \infty .
\end{cases}
$$
It is clear that $\| \bm{x} \|_p$ won't be changed after permutation of entries. Therefore, for any permutation matrix $\bm{P}$,
$$
    \|\bm{Px} \|_p  = \|\bm{x} \|_p,
$$
which is what we need to prove.
\end{proof}
\end{lemma}

\begin{corollary}
Given matrix $\bm{A} \in \mathcal{M}_{m\times n} $ and two permutation matrix $\bm{P} \in \mathcal{M}_{m \times m}, \bm{Q} \in \mathcal{M}_{n \times n}$. Then,
$$
\| \bm{PAQ}\|_{p} = \|  \bm{A} \|_{p}.
$$
\end{corollary}

\begin{proof}
    By the definition of induced norm, we can get that the LHS equals
    $$
    LHS = \|\bm{A}\|_p = \sup_{x} \frac{\| \bm{A} x\|_p}{\|x\|_p} =  \sup_{x} \frac{\|\bm{AQ} x \|_p}{\|\bm{Q} x\|_p} = \sup_{x} \frac{\|\bm{AQ} x \|_p}{\|x\|_p} = \| \bm{AQ} \|_p,
    $$
    Futhermore, the RHS equals
    $$
    RHS = \| \bm{PAQ}\|_p = \sup_{x} \frac{\|\bm{PAQ} x\|_p}{\|x\|_p} = \sup_{x} \frac{\| \bm{AQ}x \|_p}{\|x \|_p} = \| \bm{AQ}\|_p = LHS,
    $$
    which is we need to prove.
\end{proof}



\section{Solutions}
\begin{enumerate}
    \item[3.1] {
    By equation (3.3), we can get that
    $$
         \| \bm{x} \|_{\bm{W}} = \|\bm{Wx} \|,
    $$
    where $\| \cdot \|$ is a vector norm. It is clear that $\|\cdot \|_{\bm{W}}$ meets (2), (3) of the vector norm's definition. Furthermore, we assume that
    \begin{equation*}
    \bm{Wx} = \bm{0}.  \tag{$\star$}
    \end{equation*}
    }
    Since $\bm{W}$ is non-singular, $(\star)$ is true iif. $\bm{x} = \bm{0}$. Then $\| x\|_{\bm{W}} = \|\bm{Wx} \| \geq 0$, and $\| \bm{x} \|= 0 $ iif. $\| \bm{x} \| = 0$, which meets condition (1) of vector norm's definition. Hence, we can conclude that $\| \cdot \|_{\bm{W}}$ is a vector norm.

    \item[3.2] {
        Let $\lambda$ be the eigenvalue of $\bm{A}$ and $\bm{x}$ be the eigenvector w.r.t. $\lambda$, then
        \begin{equation*}
        \bm{A} x = \lambda x, \tag{$\star$}
        \end{equation*}
        where $x \neq 0$. We can concentrate m's $x$ as a matrix $\bm{X}$ with m columns, which is 
        $$
        (x, x, \cdots, x) = \bm{X} \in \mathbb{R}^{m \times n}.
        $$
        By $(\star)$, we can get that
        \begin{equation*}
            \bm{A} (x, x, \cdots, x)= \bm{A} \bm{X}=\lambda \bm{X}, \tag{$\spadesuit$}
        \end{equation*}
        Implement the norm on both sides of $(\spadesuit)$, then
        $$
        \| \lambda \bm{X} \| = | \lambda | \| \bm{X} \| = \| \bm{AX} \| \leq \| \bm{A} \| \|\bm{X} \|,
        $$
        Note that $\bm{X} \neq \bm{0}$ and hence $| \lambda | \leq \| \bm{A} \|$ for any eigenvalue $\lambda$ of $\bm{A}$. Therefore
        $$
        \rho(\bm{A}) = \max \{ \lambda \} \leq \| \bm{A} \|.
        $$
        which is exactly we want to prove.
    }
    
    \item[3.3 (a)] {
        Assume that $\bm{x} = \{x_1, \cdots, x_n \}$, then
        $$
        \| \bm{x}\| = \max_i \{x_i\} = \sqrt{x^2_{\max}} \leq \sqrt{x_1^2 + \cdots + x_m^2 } = \| \bm{x} \|_2
        $$
    }

    \item[3.3 (b)] {
        $$
        \| \bm{x}\|_2 = \sqrt{x_1^2 + \cdots + x_m^2} \leq \sqrt{x^2_{\max} + \cdots + x_{\max}^2} = \sqrt{m} \| \bm{x} \|_{\infty}.
        $$
    }

    \item[3.3 (c)] {
        $$
        \begin{aligned}
        \|\bm{A}\|_{\infty} & = \sup \frac{\| \bm{Ax} \|_{\infty}}{\|\bm{x}\|_{\infty}} \\
        & \leq \sup \frac{\| \bm{Ax} \|_{2}}{\|\bm{x}\|_{\infty}} \\
        & \leq \sup \frac{\| \bm{Ax} \|_{2}}{\|\bm{x}\|_{2}/\sqrt{n}} \\
        & = \sqrt{n} \| \bm{A} \|_2
        \end{aligned}
        $$
    }

    \item[3.4] {
        We can divide the procedure into 2 steps. Firstly, we get the certain rows: W.L.O.G., assume that we need
        the first $m$ rows of matrix $\bm{A}$, that is
        $$
        \begin{bmatrix}
            \bm{e}^T_1 \\
            \bm{e}^T_2 \\
            \vdots \\
            \bm{e}^T_m
        \end{bmatrix} \bm{A} = \bm{P},
        $$
        where $\bm{e}_i$ is the unit vector whose i-th entry is $1$, and $\bm{B}$ is the submatrix contains the first $m$ rows of $\bm{A}$. Then
        we get the certain columns from $\bm{B}$. W.L.O.G. , assume that we need the first $n$ columns of matrix $\bm{A}$, that is  
        $$
        \bm{P} [\bm{e}_1, \bm{e}_2 ,\cdots, \bm{e}_n ] = \bm{B},
        $$
        Hence,
        $$
        \bm{B} = \begin{bmatrix}
            \bm{e}^T_1 \\
            \bm{e}^T_2 \\
            \vdots \\
            \bm{e}^T_m
        \end{bmatrix} A [\bm{e}_1, \bm{e}_2 ,\cdots, \bm{e}_n ].
        $$
        Denote $ [\bm{e}^T_1, \bm{e}^T_2 ,\cdots, \bm{e}^T_m ]^{T} $ by $\bm{E}$, $[\bm{e}_1, \bm{e}_2 ,\cdots, \bm{e}_n ]$ by $\bm{F}$. Then we have that
        $$
        \| \bm{B}\| \leq \|\bm{EAF} \| \leq \| \bm{E} \| \| \bm{A} \| \| \bm{F} \|.
        $$
        It is clear that $\| \bm{E}\| = \| \bm{F} \| = 1$, then we can get that
        $$
        \| \bm{B} \| \leq \| \bm{A} \|.
        $$

        \begin{remark}
            We give another (non-rigorous enough for $p$ is odd mumber) proof here, assume that 
            $$
            \bm{A} = \begin{bmatrix}
                \bm{B} & \bm{T} \\
                \bm{Y} & \bm{Z}
            \end{bmatrix}
            $$
            then we can get that 
            $$
            \begin{aligned}
           \| \bm{B} \|_p = \sup_{\| \bm{x} \|_p = 1} \| \bm{Bx} \|_p & \leq \sup_{\| \bm{x} \|_p = 1} \begin{bmatrix}
            \bm{B} \\ \bm{Y}
           \end{bmatrix} \bm{x} \|_p \\ 
           & = \sup_{\| \bm{x} \| = 1} \| \bm{A} \begin{bmatrix}
            \bm{x} \\ \bm{0}
           \end{bmatrix}\|_p & \leq \sup_{\| \bm{u}\|_p = 1} \| \bm{Au}\|_{p} = \| \bm{A} \|_p.
            \end{aligned}
            $$
        \end{remark}
    }   
    \item[3.5] {
        Since
        $$
       \| \bm{u} \bm{v}^{\star}\|_F = tr (\bm{v} \bm{u}^{\star} \bm{u} \bm{v}^{\star}) = \| u\|_2 tr(\bm{v} \bm{v}^{\star}) = \| \bm{u}\|_2 \| \bm{v} \|_F.
        $$
        we can get that $\| \bm{u} \bm{v}^{\star}\|_F = \| \bm{u}\|_F \| \bm{v} \|_F$ is not true.
    }

    \item[3.6(a)] {
    We can prove by verifying these properties:
        \begin{enumerate}
            \item[(i)] $ \|\bm{x}\|^{\prime} = 0 \Leftrightarrow | \bm{y}^{\star} \bm{x} | = 0 ( \forall \| \bm{y} \| = 1) \Leftrightarrow \bm{x} = \bm{0}$.
            \item[(ii)] $ \| \alpha \bm{x} \|^{\prime} = \sup_{\| \bm{y} \| = 1} | \bm{y}^{\star} \alpha \bm{x} | = |\alpha| \sup_{\| \bm{y} \|  = 1}  |\bm{y}^{\star}  \bm{x} |  = |\alpha| \| \bm{x}\|^{\prime}$.
            \item[(iii)] {
                $$
                \begin{aligned}
                    \| \bm{x + z}\|^{\prime} & = \sup_{\| \bm{y} \| = 1} | \bm{y}^{\star} \bm{x} + \bm{y}^{\star} \bm{z}| \\ 
                                            & \leq \sup_{\| \bm{y} \| = 1} ( |\bm{y}^{\star} \bm{x} | + | \bm{y}^{\star} \bm{z} |) \\
                                            & \leq \sup_{\| \bm{y}_1 \| = 1} | \bm{y}_1^\star \bm{x} | + \sup_{\| \bm{y}_2\| = 1} \sup | \bm{y}_2^{\star} \bm{z} | \\
                                            & = \| \bm{x}\|^{\prime} + \| \bm{z}\|^\prime.
                \end{aligned}
                $$
            }
        \end{enumerate}
        then we can get that $\| \cdot \|^\prime$ is a norm.
    }

    \item[3.6(b)] {
        Assume that $\bm{B} = \bm{y} \bm{z}^{\star}$ such that
        $$
        \begin{cases}
            \bm{Bx} = \bm{y} \\
            \| \bm{B}\| = 1.
        \end{cases}
        $$
        which is equivalent to $\exists \bm{z} \in \mathbb{C}^m$ such that
        $$
        \begin{cases}
            \bm{z}^{\star} \bm{x} = 1 \\
            \| \bm{z}\|^{\prime} = 1.
        \end{cases}
        $$
        by lemma, we can get that $\exists \bm{u} \in \mathbb{C}^{m}$ such that
        $$
        | \bm{u}^\star \bm{x} | = \| \bm{u} \|^\prime \| \bm{x} \| = \| \bm{u} \|^{\prime},
        $$
        let $\alpha =  \bm{u}^\star \bm{x}$, we can denote $\bm{z}$ by $\bm{z} = (1/\alpha) \bm{u}$, then
        \begin{itemize}
            \item $ \| \bm{z} \|^{\prime} = \| \frac{1}{\alpha} \bm{u}\|^{\prime} = \frac{1}{|\alpha|} \| \bm{u} \|^{\prime} = 1, ~ \checkmark$
            \item $ \bm{z}^{\star} \bm{x} = \frac{1}{\alpha} \bm{u}^\star \bm{x} = 1, ~\checkmark$
        \end{itemize}
        Hence, $\bm{z}$ is exactly what we need.
    }
\end{enumerate}
